% !TeX program = xelatex
% !TeX encoding = UTF-8
% !TeX root    = listings.tex
\documentclass{../mirea-prog-lang}

\usepackage{hyperref}
\hypersetup{pdftitle={Демонстрация пакета Listings в mirea-prog-lang}, pdfauthor={В. С. Верхотуров}}

\usepackage{listings}
\usepackage{xcolor}
\lstset{basicstyle=\footnotesize, breaklines=true, numbers=left, captionpos=t, showstringspaces=false, commentstyle=\color{teal}, stringstyle=\color{red}, keywordstyle=\color{violet}}  % Настройки, применяемые ко всем листингам

\usepackage{caption}
\captionsetup[lstlisting]{justification=raggedright, singlelinecheck=false}


\title{Демонстрация пакета listings\footnote{\url{https://github.com/ValeryVerkhoturov/mirea-kb2-programming-languages}}}
\date{\today}

\begin{document}
	
\maketitle
\addtocounter{page}{1}
	
\tableofcontents
	
\section{Документация listings}

Документацию listings см. в~\url{https://ctan.org/pkg/listings}.


\section{Пример листинга для С++ ISO}

См. листинг~\ref{lst:factorial}.

\begin{lstlisting}[language={[ISO]{C++}}, caption={Нахождение факториала}, label=lst:factorial]
#include <iostream>
using namespace std;

int main() {
	int n;
	long double factorial = 1.0;
	
	cout << "Enter a positive integer: ";
	cin >> n;
	
	if (n < 0)
	cout << "Error! Factorial of a negative number doesn't exist.";
	else {
		for(int i = 1; i <= n; ++i) {
			factorial *= i;
		}
		cout << "Factorial of " << n << " = " << factorial;    
	}
	
	return 0;
}
\end{lstlisting}


\end{document}