% Compiler: XeLaTeX
\documentclass[14pt, a4paper, titlepage]{extarticle}

\usepackage[english,main=russian]{babel}
\usepackage{fontspec}
\setmainfont{Times New Roman} % Если возникают проблемы при компиляции с данной строкой, необходимо установить необходимый 
\usepackage{newtxmath} % Поменять гарнитуру в фомулах на Times New Roman

\usepackage[left=30mm, right=15mm, top=20mm, bottom=20mm]{geometry}

\usepackage{indentfirst} % Красная строка у первого абзаца раздела

\usepackage{graphicx}

\parindent=1.25cm % Размер красной строки

\parskip=6pt % Отступ между абзацами

\righthyphenmin=2 % Разрешить переносить слоги в 2 буквы (стандартное значение 3)

\linespread{1.3} % полуторный межстрочный интервал

\usepackage{tocbibind} % Добавить раздел оглавление в оглавление

% Настройка заголовка оглавления
\addto\captionsrussian{\renewcommand{\contentsname}{Оглавление}}


\usepackage{tocloft}
% Формат оглавления
\renewcommand\cfttoctitlefont{\hfill\fontsize{16pt}{16pt}\selectfont\bfseries\MakeUppercase}
\renewcommand\cftaftertoctitle{\hfill\hfill}

\renewcommand{\cftsecleader}{\cftdotfill{\cftdotsep}} % Добавить точки у разделов в оглавлении

% Настройка раздела, подраздела, подподраздела
\usepackage{titlesec}
\titleformat{\section}{\filcenter\fontsize{16pt}{16pt}\selectfont\bfseries\uppercase}{\thesection}{.2em}{}
\titleformat{\subsection}{\filcenter\bfseries}{\thesubsection}{.2em}{}
\titleformat{\subsubsection}{\filcenter\bfseries}{\thesubsubsection}{.2em}{}

\AddToHook{cmd/section/before}{\clearpage} % Начинать раздел с новой страницы

\renewenvironment{abstract}{\clearpage\section*{\MakeUppercase{\abstractname}}}{}

\labelwidth=1.25cm % Горизонтальный отступ у элемента списка

% Ненумерованные списки разной вложенности
\renewcommand\labelitemi{--}
\renewcommand\labelitemii{--}
\renewcommand\labelitemiii{--}
\renewcommand\labelitemiv{--}

% Нумерованные списки разной вложенности
\renewcommand\labelenumi{\arabic{enumi})}
\renewcommand\labelenumii{\asbuk{enumii})}
\renewcommand\labelenumiii{\arabic{enumiii})}
\renewcommand\labelenumiv{\asbuk{enumiv})}

\makeatletter
% Буквы для нумерации списка (исключены ё, з, щ, ч, ъ, ы, ь)
% Подробнее https://ctan.math.illinois.edu/macros/latex/required/babel/contrib/russian/russianb.pdf 
\def\russian@alph#1{\ifcase#1\or
    а\or б\or в\or г\or д\or е\or ж\or
    и\or к\or л\or м\or н\or о\or п\or 
    р\or с\or т\or у\or ф\or х\or ц\or 
    ш\or э\or ю\or я\else\@ctrerr\fi}
\def\russian@Alph#1{\ifcase#1\or
    А\or Б\or В\or Г\or Д\or Е\or Ж\or
    И\or К\or Л\or М\or Н\or О\or П\or 
    Р\or С\or Т\or У\or Ф\or Х\or Ц\or 
    Ш\or Э\or Ю\or Я\else\@ctrerr\fi}

\patchcmd{\l@section}{#1}{\textnormal{\uppercase{#1}}}{}{} % Разделы в оглавлении без выделения жирным, в верхнем регистре
\patchcmd{\l@section}{#2}{\textnormal{#2}}{}{} % Страницы без выделения жирным

%\patchcmd{\appendix}{\@Alph}{\MakeUppercase{\appendixname} \russian@Alph}{}{} % Нумерация приложения

\apptocmd{\appendix}{
    \renewcommand{\thesection}{\Asbuk{section}}
    \titleformat{\section}[display]{\filcenter\fontsize{16pt}{16pt}\selectfont\bfseries}{\MakeUppercase{\appendixname}~\thesection}{0pt}{}{}{} % Изменение формата раздела приложения
    
    \setcounter{secnumdepth}{0}
    \let\oldsec\section
    \renewcommand{\section}{\addtocounter{section}{1}\oldsec[\appendixname~\thesection]}
    %\renewcommand\section[1][]{\@startsection{section}{1}{\z@}{-3.5ex \@plus -1ex \@minus -.2ex}{2.3ex \@plus.2ex}{\filcenter\fontsize{16pt}{16pt}\selectfont\bfseries}[]{##1}}
}
\makeatother

\usepackage[labelsep=endash]{caption} % Настройка пунктуации
\captionsetup[table]{justification=raggedright, singlelinecheck=false} % Выравнивание по левому краю надписи таблицы

\addto\captionsrussian{\renewcommand{\figurename}{Рисунок}} % Переопределение caption из babel

% Пакет делает ссылки кликабельными и дает возможность добавить метаданные .pdf документа
\usepackage{hyperref}
% Метаданные .pdf документа, отключение подсветки ссылок
\hypersetup{pdftitle={Языки программирования}, pdfauthor={В. С. Верхотуров}, colorlinks=false, pdfborder={0 0 0}}

\usepackage{csquotes} % используется biblatex
\usepackage[
    backend=biber,
    bibstyle=gost-numeric,
    language=auto,
    autolang=other,
    sorting=none,
]{biblatex}

% Перечисление использованных источников в biblatex
% Подробнее https://www.ctan.org/pkg/biblatex-gost
\begin{filecontents*}[overwrite]{biblio.bib}
    @book{bib:recomendations,
      author = {Мерсов, А. А. and Русаков, А. М. and Филатов, В. В.},
      year = {2022},
      title = {Методические рекомендации по выполнению курсовой работы по дисциплине <<Языки программирования>>},
      publisher = {МИРЭА --- Российский технологический университет},
      chapter=3
    }
    @online{bib:symbol,
      author = {РТУ~МИРЭА},
      title = {Символика Университета},
      year = 2022,
      url = {https://www.mirea.ru/mediapage/the-symbolism-of-the-university/},
      urldate = {2022-05-31}
    }
    @online{bib:githubrepo,
      author = {Верхотуров, В. С.},
      title = {LaTeX шаблон для курсовой работы по дисциплине <<Языки программирования>> РТУ МИРЭА},
      year = 2022,
      url = {https://github.com/ValeryVerkhoturov/mirea-kb2-programming-languages},
      urldate = {2022-05-29}
    }
\end{filecontents*}
\addbibresource{biblio.bib}

\begin{document}

\begin{titlepage}
    \begin{center}
        \includegraphics[width=2.5cm]{MIREA_Gerb_Black.eps} \\
        МИНОБРНАУКИ РОССИИ \\ 
        Федеральное государственное бюджетное образовательное учреждение высшего образования \\
        \textbf{<<МИРЭА --- Российский технологический университет>>} \\
        \textbf{РТУ МИРЭА}
        \rule{\textwidth}{1pt}
        Институт кибербезопасности и цифровых технологий \\
        Кафедра КБ-14 <<Цифровые технологии обработки данных>> \\ \bigskip
        \fontsize{16pt}{16pt}\selectfont
        \textbf{КУРСОВОЙ ПРОЕКТ (РАБОТА)} \\ \bigskip \bigskip
        по дисциплине: \textbf{<<Языки программирования>>} \\ \bigskip \bigskip
        \textbf{Тема курсового проекта (работы):} \\
        <<Разработка файловой базы данных студентов на основе принципов ООП>>
    \end{center}
    \noindent\textbf{Студент группы} {\fontsize{16pt}{16pt}\selectfont БСБО-05-20 Верхотуров Валерий Сергеевич}\par
    \noindent\textbf{Руководитель курсового проекта (работы)} {\fontsize{16pt}{16pt}\selectfont Должность, звание, ученая степень}\par
    \noindent\textbf{Рецензент} {\fontsize{16pt}{16pt}\selectfont Должность, звание, ученая степень}\par
    \centering\vfill 
    Москва \the\year{} г.
\end{titlepage}

\begin{abstract}
    Вы читаете документ, имеющий настройки, соответствующие главе учебно-методического пособия~\cite{bib:recomendations}, и скомпилированный системой компьютерной верстки Xe\LaTeX{} из дистрибутива TeX Live. Ссылки кликабельны.
    
    В этом документ использованы только пакеты для форматирования (исключение --- graphicx) и библиографии. Переопределено окружение abstract, новые макрокоманды не объявлены. В родительской директории этого \verb".tex" файла должен лежать файл герба (черно-белый) mirea\_coat\_of\_arms.png~\cite{bib:symbol}. Настройки форматирования, библиографический список находятся в преамбуле (при удалении всего содержимого в окружении document настройки форматирования не изменятся).
    
    При использовании \url{overleaf.com} для компиляции документа (убедитесь, что в опциях проекта стоит компилятор XeLaTeX). Замечания о расхождении с главой учебно-методического пособия~\cite{bib:recomendations} писать в Issues репозитория~\cite{bib:githubrepo}.
    
    \subsection*{.tex в .docx}
    
    Раздел 1.1 СМКО МИРЭА 7.5.1/03.П.69-16 рекомендует использовать текстовый редактор, обеспечивающий корректное сохранение или экспорт документа в .doc (.docx). .tex не может быть напрямую экспортирован в .doc (.docx). Возможно скомпилировать .pdf, сохранить в Google Документы и экспортировать в .docx. Математические формулы экспортируются некорректно, редактировать такой документ с сохранением форматирования сложно.
    
    \subsection*{Титульная страница}
    
    Рекомендуется использовать выданную(-ые) преподавателем ти\-туль\-ну\-ю(-ые) страницу(-ы), экспортировать в .pdf и включить в этот документ (например, с помощью пакета pdfpages).
    
    При использовании своей титульной страницы необходимо удалить окружение titlepage: \\
    \verb"\begin{titlepage} ... \end{titlepage}" \\
    класс документа поменять с \\ 
    \verb"\documentclass[14pt, a4paper, titlepage]{extarticle}" \\ 
    на \\ 
    \verb"\documentclass[14pt, a4paper]{extarticle}".
    
    \subsection*{Использование графики}
    
    Для графики уже включен пакет \texttt{graphicx}, чтобы вставить герб на титульную страницу и в рис.~\ref{fig:test_label}.

    \subsection*{Окружение \texttt{abstract}}
    
    В случае отсутствия аннотации в работе удалить окружение abstract: \\\verb"\begin{abstract} ... \end{abstract}".
    
    Окружение abstract можно заменить на \\ \verb"\section*{" Аннотация\verb"}" \\ и получить визуально одинаковый результат.
    
    \subsection*{Приложение}
    
    Перед началом приложений необходимо написать команду \verb"\appendix" (стандартное поведение класса документа article).
    
    Раздел приложения: \verb"\section[]{\\"Название моего приложения\verb"}".
    
    \subsection*{Метаданные .pdf}
    
    Не забудьте в преамбуле в команде \verb"\hypersetup" поменять значение полей \verb"pdftitle={"Название моего документа\verb"}", \verb"pdfauthor={"Мое имя\verb"}".
\end{abstract}

\clearpage
\tableofcontents

\section{Тесты}

\subsection{Перечни}
 
\begin{itemize}
    \item первый уровень вложенности
    \begin{itemize}
        \item второй
        \begin{itemize}
            \item третий
            \begin{itemize}
                \item четвертый
            \end{itemize}
        \end{itemize}
    \end{itemize}
\end{itemize}
 
\begin{enumerate}
    \item первый уровень вложенности
    \begin{enumerate}
        \item второй
        \begin{enumerate}
            \item третий
            \begin{enumerate}
                \item четвертый
                \item должно быть б
                \item в
                \item г
                \item д
                \item е
                \item ж
                \item и
            \end{enumerate}
        \end{enumerate}
    \end{enumerate}
\end{enumerate}

\subsection{Плавающие элементы}
 
\begin{table}[htb]
    \caption{Подпись выше таблицы слева}
    \begin{tabular}{ |c|c|c| } 
        \hline
        ячейка 1 & ячейка 2 & ячейка 3 \\ \hline
        ячейка 4 & ячейка 5 & ячейка 6 \\ \hline
    \end{tabular}
    \label{tab:test_label}
\end{table}
 
\begin{figure}[htb]
    \centering
    \includegraphics[width=.5\textwidth]{MIREA_Gerb_Black.eps}
    \caption{Подпись ниже рисунка по центру}
    \label{fig:test_label}
\end{figure}
 
См. таблицу~\ref{tab:test_label} на стр.~\pageref{tab:test_label}, рис.~\ref{fig:test_label} на стр.~\pageref{fig:test_label}.

\subsection{Формулы}

\begin{equation}\label{eq:test_label}
    (\text{минус}\,a)\times b=c
\end{equation}

Представлена формула~\ref{eq:test_label}.

\subsection{Цитирование}

Данный \verb".tex" документ создан на основе рекомендаций~\parencite{bib:recomendations}.

\subsection{Ссылка на приложение}

Ссылка на приложение выглядит так~\ref{appendix:test_label}.


\clearpage
\addcontentsline{toc}{section}{Список использованных источников}
\printbibliography[title={Список использованных источников}]


\appendix

\section{Название первого приложения}
\label{appendix:test_label}

За порядком приложений необходимо следить самостоятельно.

\section{Название второго приложения}

\end{document}
