\documentclass[14pt, a4paper, titlepage]{extarticle}

\usepackage[english,main=russian]{babel}
\usepackage{fontspec}
\setmainfont{Times New Roman} % Если возникают проблемы при компиляции с данной строкой, необходимо установить необходимый 
\usepackage{newtxmath} % Поменять гарнитуру в фомулах на Times New Roman

\usepackage[left=30mm, right=15mm, top=20mm, bottom=20mm]{geometry}

\usepackage{indentfirst} % Красная строка у первого абзаца раздела

\usepackage{graphicx}

\parindent=1.25cm % Размер красной строки

\parskip=0pt % Отступ между абзацами

\righthyphenmin=2 % Разрешить переносить слоги в 2 буквы (стандартное значение 3)

\linespread{1.3} % полуторный межстрочный интервал

\usepackage{tocbibind} % Добавить раздел оглавление в оглавление

% Настройка заголовка оглавления
\addto\captionsrussian{\renewcommand{\contentsname}{Оглавление}}

\usepackage[normalem]{ulem} % underline some lines


\usepackage{tocloft}
% Формат оглавления
\renewcommand\cfttoctitlefont{\hfill\fontsize{16pt}{16pt}\selectfont\bfseries\MakeUppercase}
\renewcommand\cftaftertoctitle{\hfill\hfill}

\renewcommand{\cftsecleader}{\cftdotfill{\cftdotsep}} % Добавить точки у разделов в оглавлении

% Настройка раздела, подраздела, подподраздела
\usepackage{titlesec}
\titleformat{\section}{\parskip=6pt\filcenter\fontsize{16pt}{16pt}\selectfont\bfseries\uppercase}{\thesection}{.5em}{}
\titleformat{\subsection}{\filcenter\bfseries}{\thesubsection}{.5em}{}
\titleformat{\subsubsection}{\filcenter\bfseries}{\thesubsubsection}{.5em}{}

\AddToHook{cmd/section/before}{\clearpage} % Начинать раздел с новой страницы

\renewenvironment{abstract}{\clearpage\section*{\MakeUppercase{\abstractname}}}{\clearpage}

\labelwidth=1.25cm % Горизонтальный отступ у элемента списка

% Ненумерованные списки разной вложенности
\renewcommand\labelitemi{--}
\renewcommand\labelitemii{--}
\renewcommand\labelitemiii{--}
\renewcommand\labelitemiv{--}

% Нумерованные списки разной вложенности
\renewcommand\labelenumi{\arabic{enumi})}
\renewcommand\labelenumii{\asbuk{enumii})}
\renewcommand\labelenumiii{\arabic{enumiii})}
\renewcommand\labelenumiv{\asbuk{enumiv})}

\makeatletter
% Буквы для нумерации списка (исключены ё, з, щ, ч, ъ, ы, ь)
% Подробнее https://ctan.math.illinois.edu/macros/latex/required/babel/contrib/russian/russianb.pdf 
\def\russian@alph#1{\ifcase#1\or
    а\or б\or в\or г\or д\or е\or ж\or
    и\or к\or л\or м\or н\or о\or п\or 
    р\or с\or т\or у\or ф\or х\or ц\or 
    ш\or э\or ю\or я\else\@ctrerr\fi}
\def\russian@Alph#1{\ifcase#1\or
    А\or Б\or В\or Г\or Д\or Е\or Ж\or
    И\or К\or Л\or М\or Н\or О\or П\or 
    Р\or С\or Т\or У\or Ф\or Х\or Ц\or 
    Ш\or Э\or Ю\or Я\else\@ctrerr\fi}

\patchcmd{\l@section}{#1}{\textnormal{\uppercase{#1}}}{}{} % Разделы в оглавлении без выделения жирным, в верхнем регистре
\patchcmd{\l@section}{#2}{\textnormal{#2}}{}{} % Страницы без выделения жирным

\apptocmd{\appendix}{
    \renewcommand{\thesection}{\Asbuk{section}}
    \titleformat{\section}{\filcenter\fontsize{16pt}{16pt}\selectfont\bfseries}{}{0pt}{\MakeUppercase{\appendixname}~\thesection \\}{}{} % Изменение формата раздела приложения
    
    \let\oldsec\section
    \renewcommand{\section}{
        \clearpage
        \phantomsection
        \refstepcounter{section}
        \addcontentsline{toc}{section}{\appendixname~\thesection}
        \oldsec*} % Нумерация раздела после названия
}
\makeatother

\usepackage[labelsep=endash]{caption} % Настройка пунктуации
\captionsetup[table]{justification=raggedright, singlelinecheck=false} % Выравнивание по левому краю надписи таблицы

\addto\captionsrussian{\renewcommand{\figurename}{Рисунок}} % Переопределение caption из babel

% Пакет делает ссылки кликабельными и дает возможность добавить метаданные .pdf документа
\usepackage{hyperref}
% Метаданные .pdf документа, отключение подсветки ссылок
\hypersetup{pdftitle={Языки программирования}, pdfauthor={В. С. Верхотуров}, colorlinks=false, pdfborder={0 0 0}}

\usepackage{csquotes} % используется biblatex
\usepackage[
    backend=biber,
    bibstyle=gost-numeric,
    language=auto,
    autolang=other,
    sorting=none,
]{biblatex}

\pretocmd{\printbibliography}{
    \clearpage
    \addcontentsline{toc}{section}{Список использованных источников}}{}{}

% В библиографии нет отступов начиная со второй строки
\defbibenvironment{bibliography}
  {\list
     {\printtext[labelnumberwidth]{%
        \printfield{labelprefix}%
        \printfield{labelnumber}}}
     {\setlength{\labelwidth}{-\labelnumberwidth}%
      \setlength{\labelsep}{\biblabelsep}%
      \setlength{\leftmargin}{0pt}%
      \setlength{\itemindent}{\bibhang}%
      \setlength{\itemsep}{\bibitemsep}%
      \setlength{\parsep}{\bibparsep}}%
      \renewcommand*{\makelabel}[1]{\hss##1}}
  {\endlist}
  {\item}

\let\oldprintbibliography\printbibliography
\renewcommand{\printbibliography}{\oldprintbibliography[title={Список использованных источников}]}

% Перечисление использованных источников в biblatex
% Подробнее https://www.ctan.org/pkg/biblatex-gost
\begin{filecontents*}[overwrite]{biblio.bib}
    @online{bib:recomendations,
      author = {Мерсов, А. А. and Русаков, А. М. and Филатов, В. В.},
      year = {2022},
      title = {Языки программирования [Электронный ресурс]: методические рекомендации по выполнению курсовой работы},
      publisher = {РТУ МИРЭА},
      url = {https://library.mirea.ru/share/4488},
      urldate = {2022-06-16}
    }
    @online{bib:symbol,
      author = {РТУ~МИРЭА},
      title = {Символика Университета},
      year = 2022,
      url = {https://www.mirea.ru/mediapage/the-symbolism-of-the-university/},
      urldate = {2022-05-31}
    }
    @online{bib:githubrepo,
      author = {Верхотуров, В. С.},
      title = {LaTeX шаблон для курсовой работы по дисциплине <<Языки программирования>> РТУ МИРЭА},
      year = 2022,
      url = {https://github.com/ValeryVerkhoturov/mirea-kb2-programming-languages},
      urldate = {2022-05-29}
    }
\end{filecontents*}
\addbibresource{biblio.bib}

\begin{document}

\begin{titlepage}
    \pagestyle{empty}
    \setlength\parindent{0pt}
    \newcommand\blankDate{\mbox{\uline{<<\qquad>>\hspace{2cm}\the\year{}~г.}}}
    \newcommand\blankLine[2]{$\underset{\text{#1}}{\text{\uline{\hfil\penalty100\hfilneg#2}}}$}
    \begin{center}
        \includegraphics[width=2.5cm]{MIREA_Gerb_Black} \par
        МИНОБРНАУКИ РОССИИ \par 
        Федеральное государственное бюджетное образовательное учреждение высшего образования \par
        \textbf{<<МИРЭА --- Российский технологический университет>>} \par
        \textbf{\fontsize{16pt}{16pt}\selectfont РТУ МИРЭА} \par
        \blankLine{(наименование института, филиала)}{Институт кибербезопасности и цифровых технологий} \par
        \blankLine{(наименование кафедры)}{Кафедра КБ-2 <<Прикладные информационные технологии>>} \par
        \begin{flushright}
            \begin{minipage}{.5\textwidth}
                \fontsize{12pt}{12pt}\selectfont
                \setlength{\parskip}{0pt}
                \centering
                Утверждаю \par
                Заведующий кафедрой КБ-2 \par
                \blankLine{(Ф.И.О.)}{Фамилия~И.О.}~\blankLine{(подпись)}{\hspace{2cm}} \par
                \blankDate
            \end{minipage}
        \end{flushright}
        {\fontsize{16pt}{16pt}\selectfont
        \textbf{КУРСОВАЯ РАБОТА}} \par
        по дисциплине \blankLine{(наименование дисциплины)}{Языки программирования}
    \end{center}
    \textbf{Тема курсовой работы \uline{\hspace{11cm}}} \bigskip\par
    Студент группы \blankLine{учебная группа, фамилия, имя, отчество студента}{БСБО-05-20 Верхотуров Валерий Сергеевич} \hfill\blankLine{подпись студента}{\hspace{3cm}} \bigskip\par
    Руководитель курсовой работы \blankLine{должность, звание, учёная степень}{\hspace{6cm}} \hfill\blankLine{подпись руководителя}{\hspace{3cm}} \bigskip\par
    Рецензент (при наличии) \blankLine{должность, звание, учёная степень}{\hspace{7cm}} \hfill\blankLine{подпись рецензента}{\hspace{3cm}} \bigskip\par
    \begin{tabular}{@{}ll}
        Работа предоставлена к защите & \blankDate \bigskip\\
        Допущен к защите & \blankDate
    \end{tabular}
    \begin{center}
        \vfill Москва~---~\the\year{}~г.
    \end{center}
    \newpage
    \textbf{Срок предоставления к защите курсовой работы до} \hfill\blankDate \par
    \textbf{Задание на курсовую работу выдал} \blankLine{(Ф.И.О. руководителя)}{\hspace{4cm}} \hfill\blankLine{(подпись руководителя)}{\hspace{3cm}} \par
    \hfill\blankDate \par
    \textbf{Задание на курсовую работу получил} \blankLine{(Ф.И.О. исполнителя)}{\hspace{3cm}} \hfill\blankLine{(подпись исполнителя)}{\hspace{3cm}} \par\bigskip
    \begin{center}
        Москва~---~\the\year{}~г.
    \end{center}
\end{titlepage}
\addtocounter{page}{2}

\begin{abstract}
    Этот документ имеет настройки, соответствующие главе 3 у\-чеб\-но-ме\-то\-ди\-чес\-ко\-го пособия~\cite{bib:recomendations}, скомпилированный системой компьютерной вёрстки XeTeX. 
    
    В документе использованы только пакеты для форматирования, пакет biblatex для форматирования библиографии и graphix для иллюстраций. В родительской директории главного .tex файла должен лежать файл чёрно-белого герба для титульной страницы \verb"MIREA_Gerb_Black" (в шаблоне используется .eps файл --- единственный векторный формат, предоставляемый на сайте вуза~\cite{bib:symbol}, по причине \href{https://www.cve.org/CVERecord?id=CVE-2013-4979}{CVE-2013-4979} также возможно использование .jpeg, .png). Настройки форматирования, список источников находятся в преамбуле (при удалении всего содержимого из окружения document настройки форматирования не изменятся).
    
    При использовании \url{overleaf.com} убедитесь, что в опциях проекта стоит компилятор XeLaTeX. 
    
    Замечания о расхождении с методическими рекомендациями~\cite{bib:recomendations} можно писать в Issues Github репозитория~\cite{bib:githubrepo}, задать вопрос в Telegram \url{https://t.me/ValerianaOfficinalis}.
    
    \subsection*{.tex в .docx}
    
    Раздел 1.1 \href{http://stavropol.mirea.ru/images/vipusk/rek.pdf}{СМКО МИРЭА 7.5.1/03.П.69-16} рекомендует использовать текстовый редактор, обеспечивающий корректное сохранение или экспорт документа в .doc (.docx). Шаблон .tex не может быть экспортирован в .doc (.docx). Возможно скомпилировать .pdf, сохранить в Google Документы и экспортировать в .docx или воспользоваться аналогичным конвертером. При этом настройки форматирования документа не сохраняются, возможен некорректный экспорт математических формул.
    
    \subsection*{Метаданные .pdf}
    
    Не забудьте в преамбуле в команде \verb"\hypersetup" поменять значение полей \verb"pdftitle={"Название моего документа\verb"}", \verb"pdfauthor={"Мое имя\verb"}".
    
\end{abstract}

\tableofcontents

\section*{Введение}
\phantomsection
\addcontentsline{toc}{section}{Введение}

Содержание введения.

\section{Структура документа}

Структура документа совместима со стандартным классом документа article.

\subsection{Титульная страница}

Рекомендуется использовать выданные преподавателем титульные страницы (например, с помощью пакета pdfpages).
    
При использованих своих титульных страниц необходимо удалить окружение titlepage:

\begin{verbatim}
\begin{titlepage}
...
\end{titlepage}
\end{verbatim}
класс документа поменять на

\begin{verbatim}
\documentclass[14pt, a4paper]{extarticle} 
\end{verbatim}

К основному тексту после титульных листов необходимо добавить кол-во вставленных страниц. Пример:
\begin{verbatim}
\begin{titlepage}
...
\end{titlepage}
\addtocounter{page}{2}
\end{verbatim}

\subsection{Аннотация}

\begin{verbatim}
\begin{abstract}
...
\end{abstract}
\end{verbatim}

\subsection{Оглавление}

\begin{verbatim}
\tableofcontents
\end{verbatim}

\subsection{Введение, раздел без нумерации}

\noindent\verb"\section*{"Введение\verb"}"\\
\verb"\phantomsection"\\
\verb"\addcontentsline{toc}{section}{"Введение\verb"}"\\
\verb"..."

\subsection{Раздел}

\begin{verbatim}
\section{...}
...
\end{verbatim}

\subsection{Подраздел}

\begin{verbatim}
\subsection{...}
...
\end{verbatim}

\subsection{Пункт}

\begin{verbatim}
\subsubsection{...}
...
\end{verbatim}

\subsubsection{Пример пункта}

Текст пункта.

\subsection{Список использованных источников}

\begin{verbatim}
\printbibliography
\end{verbatim}

Ссылка на источник:

\begin{verbatim}
\cite{...}
\end{verbatim}

Пример ссылки на источник~\cite{bib:recomendations}.

В шаблоне источники заполняются в преамбуле.

Сведения об источниках располагаются в порядке появления ссылок на источники автоматически. Источники, на которых ссылок нет, не добавляются в раздел \uppercase{<<Список использованных источников>>}.

\subsection{Приложение}

Как и в стандартных классах перед приложениями необходимо указать команду \verb"\appendix".

Пример с одним приложением:

\begin{verbatim}
\appendix
\section{...}
...
\end{verbatim}

Пример с тремя приложениями:

\begin{verbatim}
\appendix
\section{...}
...
\section{...}
...
\section{...}
...
\end{verbatim}

Ссылка на приложение

\begin{verbatim}
\ref{...}
\end{verbatim}

Пример ссылки на приложение~\ref{appendix:test_label}.

За порядком приложений необходимо следить самостоятельно.

\subsection{Список}

\begin{verbatim}
\begin{itemize}
    \item ...,
    ...
\end{itemize}
\end{verbatim}

\begin{itemize}
    \item[] Пример списка:
    \item первый уровень вложенности,
    \begin{itemize}
        \item второй,
        \begin{itemize}
            \item третий,
            \begin{itemize}
                \item четвертый.
            \end{itemize}
        \end{itemize}
    \end{itemize}
\end{itemize}

\subsection{Перечисление}

\begin{verbatim}
\begin{enumerate}
    \item ...;
    ...
\end{enumerate} 
\end{verbatim}

\begin{enumerate}
    \item[] Пример перечисления:
    \item первый уровень вложенности;
    \begin{enumerate}
        \item второй;
        \begin{enumerate}
            \item третий;
            \begin{enumerate}
                \item четвертый;
                \item б;
                \item в;
                \item г;
                \item д;
                \item е;
                \item ж;
                \item и;
                \item к;
                \item л;
                \item м;
                \item н;
                \item о;
                \item п;
                \item р;
                \item с;
                \item т;
                \item у;
                \item ф;
                \item x;
                \item ц;
                \item ш;
                \item э;
                \item ю;
                \item я, при наличии б\'ольшего кол-ва элементов компилятор выдаст ошибку.
            \end{enumerate}
        \end{enumerate}
    \end{enumerate}
\end{enumerate}

\subsection{Иллюстрация}
 
Пакет graphicx подключен. См. рисунок~\ref{fig:test_label} на стр.~\pageref{fig:test_label}.

\begin{figure}[htb]
    \centering
    \includegraphics[width=.5\textwidth]{MIREA_Gerb_Black}
    \parskip=6pt
    \caption{Подпись ниже рисунка по центру}
    \label{fig:test_label}
\end{figure}

Обратите внимание, что окружение figure является \emph{плавающим}, и иллюстрация может появиться не там, где Вы ожидаете. Для размещения иллюстрации в конкретное место необходимо воспользоваться опцией H из пакета float (не подключен).

\subsection{Таблица}

См. таблицу~\ref{tab:test_label} на стр.~\pageref{tab:test_label}.

\begin{table}[htb]
    \caption{Подпись выше таблицы слева}
    \centering
    \begin{tabular}{ |c|c|c|c|c| } 
        \hline
        ячейка 1 & ячейка 2 & ячейка 3 & ячейка 4 & ячейка 5 \\ \hline
        ячейка 6 & ячейка 7 & ячейка 8 & ячейка 9 & ячейка 10 \\ \hline
    \end{tabular}
    \label{tab:test_label}
\end{table}

Обратите внимание, что окружение table является \emph{плавающим}, и таблица может появиться не там, где Вы ожидаете. Для размещения таблицы в конкретное место необходимо воспользоваться опцией H из пакета float (не подключен).

\subsection{Уравнение и формула}

См. формулу~\ref{eq:test_label}.

\begin{equation}\label{eq:test_label}
    \text{минус}\,a\times b=c ,
\end{equation}
где $a$ --- первая переменная; \\
$b$ --- вторая переменная; \\
$c$ --- третья переменная.

\printbibliography

\appendix

\section{Название первого приложения}
\label{appendix:test_label}

Содержание первого приложения.

\section{Название второго приложения}

Содержание второго приложения.

\end{document}
