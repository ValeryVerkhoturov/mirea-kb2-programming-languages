% Compiler: XeLaTeX
\documentclass[14pt, a4paper, titlepage]{extarticle}

\usepackage[english,main=russian]{babel}
\usepackage{fontspec}
\setmainfont{Times New Roman} % Если возникают проблемы при компиляции с данной строкой, необходимо установить необходимый шрифт

\usepackage[left=30mm, right=15mm, top=20mm, bottom=20mm]{geometry}

\usepackage{csquotes} % используется biblatex
\usepackage[
    backend=biber,
    bibstyle=gost-numeric,
    language=auto,
    autolang=other,
    sorting=none,
]{biblatex}

% Перечисление использованных источников в biblatex
\begin{filecontents*}[overwrite]{biblio.bib}
    @book{bib:recomendations,
      author = {Мерсов, А. А. and Русаков, А. М. and Филатов, В. В.},
      year = {2022},
      title = {Методические рекомендации по выполнению курсовой работы по дисциплине <<Языки программирования>>},
      publisher = {МИРЭА --- Российский технологический университет},
    }
    @online{bib:prince,
      author = {Сент-Экзюпери, Антуан},
      title = {Маленький принц},
      year = 1943,
      url = {http://www.chertyaka.ru/detskie_skazki/exuperi/malenkii_princ8.php},
      urldate = {2022-05-29}
}
\end{filecontents*}
\addbibresource{biblio.bib}

\usepackage{hyperref}
\hypersetup{pdftitle={Заголовок}, pdfauthor={В. С. Верхотуров}, colorlinks=false, pdfborder={0 0 0}}

\usepackage{indentfirst} % Красная строка у первого абзаца раздела

\parindent=1.25cm % Размер красной строки

\parskip=6pt % Отступ между абзацами

\righthyphenmin=2 % Разрешить переносить слоги в 2 буквы (стандартное значение 3)

\linespread{1.3} % Межстрочный интервал 1.5

\usepackage{tocbibind} % Добавить раздел содержание в содержание

% Настройка раздела, подраздела, подподраздела
\usepackage{titlesec}
\titleformat{\section}{\clearpage\filcenter\fontsize{16pt}{16pt}\selectfont\bfseries\uppercase}{\thesection}{.2em}{}
\titleformat{\subsection}{\filcenter\bfseries}{\thesubsection}{.2em}{}
\titleformat{\subsubsection}{\filcenter\bfseries}{\thesubsubsection}{.2em}{}

\labelwidth=1.25cm % Горизонтальный отступ у элемента списка

% Ненумерованные списки разной вложенности
\renewcommand\labelitemi{--}
\renewcommand\labelitemii{--}
\renewcommand\labelitemiii{--}
\renewcommand\labelitemiv{--}

% Нумерованные списки разной вложенности
\renewcommand\labelenumi{\arabic{enumi})}
\renewcommand\labelenumii{\asbuk{enumii})}
\renewcommand\labelenumiii{\arabic{enumiii})}
\renewcommand\labelenumiv{\asbuk{enumiv})}

% Буквы для нумерации списка (исключены ё, з, щ, ч, ъ, ы, ь
% Подробнее https://ctan.math.illinois.edu/macros/latex/required/babel/contrib/russian/russianb.pdf 
\makeatletter
\def\russian@alph#1{\ifcase#1\or
    а\or б\or в\or г\or д\or е\or ж\or
    и\or к\or л\or м\or н\or о\or п\or 
    р\or с\or т\or у\or ф\or х\or ц\or 
    ш\or э\or ю\or я\else\@ctrerr\fi}
\makeatother

\usepackage[labelsep=endash]{caption} % Настройка пунктуации
\captionsetup[table]{justification=raggedright, singlelinecheck=false} % Выравнивание по левому краю надписи таблицы

\addto\captionsrussian{\renewcommand{\figurename}{Рисунок}} % Переопределение caption из babel

% Пакет делает ссылки кликабельными и дает возможность добавить метаданные .pdf документа
\usepackage{hyperref}
% Метаданные .pdf документа, отключение подсветки ссылок
\hypersetup{pdftitle={Языки программирования}, pdfauthor={В. С. Верхотуров}, colorlinks=false, pdfborder={0 0 0}}


\begin{document}

% Желательно использовать выданный преподавателем титульник. См. как https://www.ctan.org/pkg/pdfpages
% При использовании своего титульника удалить окружение titlepage (\begin{titlepage}...\end{titlepage}), класс документа поменять с \documentclass[14pt, a4paper, titlepage]{extarticle} на \documentclass[14pt, a4paper]{extarticle}
\begin{titlepage}
    \begin{center}
        \vspace*{2cm} % место для герба
        МИНОБРНАУКИ РОССИИ\par 
        Федеральное государственное бюджетное образовательное учреждение высшего образования\par
        \textbf{<<МИРЭА --- Российский технологический университет>>}\par
        \textbf{РТУ МИРЭА}
        \rule{\textwidth}{1pt}
        Институт кибербезопасности и цифровых технологий\par
        Кафедра КБ-14 <<Цифровые технологии обработки данных>>\par
        \fontsize{16pt}{16pt}\selectfont
        \textbf{КУРСОВОЙ ПРОЕКТ (РАБОТА)} \par \bigskip по дисциплине: \textbf{<<Языки программирования>>}\par
        \bigskip 
        \textbf{Тема курсового проекта (работы):} \par
        <<Разработка файловой базы данных студентов на основе принципов ООП>>\par
    \end{center}
    \noindent\textbf{Студент группы} {\fontsize{16pt}{16pt}\selectfont БСБО-05-20 Верхотуров Валерий Сергеевич}\par
    \noindent\textbf{Руководитель курсового проекта (работы)} {\fontsize{16pt}{16pt}\selectfont Должность, звание, ученая степень}\par
    \noindent\textbf{Рецензент} {\fontsize{16pt}{16pt}\selectfont Должность, звание, ученая степень}\par
    \centering\vfill 
    Москва \the\year{} г.
\end{titlepage}

\tableofcontents
\newpage

\section{Пример вёрстки}

Очень скоро я лучше узнал этот цветок. На планете Маленького принца всегда росли простые, скромные цветы~--- у них было мало лепестков, они занимали совсем мало места и никого не беспокоили. Они раскрывались поутру в траве и под вечер увядали. А этот пророс однажды из зерна, занесенного неведомо откуда, и Маленький принц не сводил глаз с крохотного ростка, не похожего на все остальные ростки и былинки. Вдруг это какая-нибудь новая разновидность баобаба? Но кустик быстро перестал тянуться ввысь, и на нем появился бутон. Маленький принц никогда еще не видал таких огромных бутонов и предчувствовал, что увидит чудо. А неведомая гостья, скрытая в стенах своей зеленой комнатки, все готовилась, все прихорашивалась. Она заботливо подбирала краски. Она наряжалась неторопливо, один за другим примеряя лепестки. Она не желала явиться на свет встрепанная, точно какой-нибудь мак. Она хотела показаться во всем блеске своей красоты. Да, это была ужасная кокетка! Таинственные приготовления длились день за днем. И вот однажды утром, едва взошло солнце, лепестки раскрылись.\cite{bib:prince}

\subsection{Подраздел}

И красавица, которая столько трудов положила, готовясь к этой минуте, сказала, позевывая:

--- Ах, я насилу проснулась\dots{} Прошу извинить\dots{} Я еще совсем растрепанная\dots{}

Маленький принц не мог сдержать восторга:

--- Как вы прекрасны!

--- Да, правда?~--- был тихий ответ.~--- И заметьте, я родилась вместе с солнцем.

Маленький принц, конечно, догадался, что удивительная гостья не страдает избытком скромности, зато она была так прекрасна, что дух захватывало!

А она вскоре заметила:

--- Кажется, пора завтракать. Будьте так добры, позаботьтесь обо мне\dots{}

Маленький принц очень смутился, разыскал лейку и полил цветок ключевой водой.

Скоро оказалось, что красавица горда и обидчива, и Маленький принц совсем с ней измучился. У нее было четыре шипа, и однажды она сказала ему:

--- Пусть приходят тигры, не боюсь я их когтей!

--- На моей планете тигры не водятся,~--- возразил Маленький принц.~--- И потом, тигры не едят траву.\cite{bib:prince}

\subsubsection{Подподраздел}

--- Я не трава,~--- тихо заметил цветок.

--- Простите меня\dots{}

--- Нет, тигры мне не страшны, но я ужасно боюсь сквозняков. У вас нет ширмы?

--- Растение, а боится сквозняков\dots{} очень странно\dots{}~--- подумал Маленький принц.~--- Какой трудный характер у этого цветка.

--- Когда настанет вечер, накройте меня колпаком. У вас тут слишком холодно. Очень неуютная планета. Там, откуда я прибыла\dots{}

Она не договорила. Ведь ее занесло сюда, когда она была еще зернышком. Она ничего не могла знать о других мирах. Глупо лгать, когда тебя так легко уличить! Красавица смутилась, потом кашлянула раз-другой, чтобы Маленький принц почувствовал, как он перед нею виноват:

--- Где же ширма?

--- Я хотел пойти за ней, но не мог же я вас не дослушать!

Тогда она закашляла сильнее: пускай его все-таки помучит совесть!

Хотя Маленький принц и полюбил прекрасный цветок, и рад был ему служить, но вскоре в душе его пробудились сомнения. Пустые слова он принимал близко к сердцу и стал чувствовать себя очень несчастным.

--- Напрасно я ее слушал,~--- доверчиво сказал он мне однажды.~--- Никогда не надо слушать, что говорят цветы. Надо просто смотреть на них и дышать их ароматом. Мой цветок напоил благоуханием всю мою планету, а я не умел ему радоваться. Эти разговоры о когтях и тиграх\dots{} Они должны бы меня растрогать, а я разозлился\dots{}

И еще он признался:

--- Ничего я тогда не понимал! Надо было судить не по словам, а по делам. Она дарила мне свой аромат, озаряла мою жизнь. Я не должен был бежать. За этими жалкими хитростями и уловками надо было угадать нежность. Цветы так непоследовательны! Но я был слишком молод, я еще не умел любить.\cite{bib:prince}



\section{Тесты}

\subsection{Списки}
 
\begin{itemize}
    \item первый уровень вложенности
    \begin{itemize}
        \item второй
        \begin{itemize}
            \item третий
            \begin{itemize}
                \item четвертый
            \end{itemize}
        \end{itemize}
    \end{itemize}
\end{itemize}
 
\begin{enumerate}
    \item первый уровень вложенности
    \begin{enumerate}
        \item второй
        \begin{enumerate}
            \item третий
            \begin{enumerate}
                \item четвертый
                \item должно быть б
                \item в
                \item г
                \item д
                \item е
                \item ж
                \item и
            \end{enumerate}
        \end{enumerate}
    \end{enumerate}
\end{enumerate}

\subsection{Плавающие элементы}
 
\begin{table}[htb]
    \caption{Подпись выше таблицы слева}
    \begin{tabular}{ |c|c|c| } 
        \hline
        ячейка 1 & ячейка 2 & ячейка 3 \\ \hline
        ячейка 4 & ячейка 5 & ячейка 6 \\ \hline
    \end{tabular}
    \label{tab:test_label}
\end{table}
 
\begin{figure}[htb]
    \centering
    Мой рисунок
    \caption{Подпись ниже рисунка по центру}
    \label{fig:test_label}
\end{figure}
 
См. таблицу~\ref{tab:test_label} на стр.~\pageref{tab:test_label}, рис.~\ref{fig:test_label} на стр.~\pageref{fig:test_label}.

\subsection{Формулы}

\begin{equation}\label{eq:test_label}
    a+b=c
\end{equation}

Представлена формула~\ref{eq:test_label}.

\subsection{Цитироваие}

Данный \verb".tex" документ создан на основе рекомендаций~\parencite{bib:recomendations}.

\clearpage
\addcontentsline{toc}{section}{Список использованных источников}
\printbibliography[title={Список использованных источников}]

\end{document}
