\documentclass[14pt, a4paper, titlepage]{extarticle}

% Пакет делает ссылки кликабельными и дает возможность добавить метаданные .pdf документа
\usepackage{hyperref}
% Метаданные .pdf документа, отключение подсветки ссылок
\hypersetup{pdftitle={Языки программирования}, pdfauthor={В. С. Верхотуров}, colorlinks=false, pdfborder={0 0 0}} % ПОМЕНЯТЬ pdftitle, pdfauthor


\usepackage[english,main=russian]{babel}
\usepackage{fontspec}
\setmainfont{Times New Roman} % Если возникают проблемы при компиляции с данной строкой, необходимо на компьютер установить  Times New Roman 
\usepackage{newtxmath} % Поменять гарнитуру в фомулах на Times New Roman

\usepackage[left=30mm, right=15mm, top=20mm, bottom=20mm]{geometry}

\usepackage{indentfirst} % Красная строка у первого абзаца раздела

\usepackage{graphicx}

\parindent=1.25cm % Размер красной строки

\parskip=0pt % Отступ между абзацами

\righthyphenmin=2 % Разрешить переносить слоги в 2 буквы (стандартное значение 3)

\linespread{1.3} % полуторный межстрочный интервал

\usepackage{tocbibind} % Добавить раздел оглавление в оглавление

% Настройка заголовка оглавления
\addto\captionsrussian{\renewcommand{\contentsname}{Оглавление}}

\usepackage[normalem]{ulem} % underline some lines


\usepackage{tocloft}
% Формат оглавления
\renewcommand\cfttoctitlefont{\hfill\fontsize{16pt}{16pt}\selectfont\bfseries\MakeUppercase}
\renewcommand\cftaftertoctitle{\hfill\hfill}

\renewcommand{\cftsecleader}{\cftdotfill{\cftdotsep}} % Добавить точки у разделов в оглавлении

\usepackage{placeins} % Команда \FloatBarrier для размещения плавающего окружения в пределах раздела, подраздела, пункта

% Настройка раздела, подраздела, подподраздела
\usepackage{titlesec}
\titleformat{\section}{\FloatBarrier\parskip=6pt\filcenter\fontsize{16pt}{16pt}\selectfont\bfseries\uppercase}{\thesection}{.5em}{}
\titleformat{\subsection}{\FloatBarrier\filcenter\bfseries}{\thesubsection}{.5em}{}
\titleformat{\subsubsection}{\FloatBarrier\filcenter\bfseries}{\thesubsubsection}{.5em}{}


\AddToHook{cmd/section/before}{\clearpage} % Начинать раздел с новой страницы

\renewenvironment{abstract}{\clearpage\section*{\MakeUppercase{\abstractname}}}{\clearpage}

\labelwidth=1.25cm % Горизонтальный отступ у элемента списка

% Ненумерованные списки разной вложенности
\renewcommand\labelitemi{--}
\renewcommand\labelitemii{--}
\renewcommand\labelitemiii{--}
\renewcommand\labelitemiv{--}

% Нумерованные списки разной вложенности
\renewcommand\labelenumi{\arabic{enumi})}
\renewcommand\labelenumii{\asbuk{enumii})}
\renewcommand\labelenumiii{\arabic{enumiii})}
\renewcommand\labelenumiv{\asbuk{enumiv})}

\makeatletter
% Буквы для нумерации списка (исключены ё, з, щ, ч, ъ, ы, ь)
% Подробнее https://ctan.math.illinois.edu/macros/latex/required/babel/contrib/russian/russianb.pdf 
\def\russian@alph#1{\ifcase#1\or
    а\or б\or в\or г\or д\or е\or ж\or
    и\or к\or л\or м\or н\or о\or п\or 
    р\or с\or т\or у\or ф\or х\or ц\or 
    ш\or э\or ю\or я\else\@ctrerr\fi}
\def\russian@Alph#1{\ifcase#1\or
    А\or Б\or В\or Г\or Д\or Е\or Ж\or
    И\or К\or Л\or М\or Н\or О\or П\or 
    Р\or С\or Т\or У\or Ф\or Х\or Ц\or 
    Ш\or Э\or Ю\or Я\else\@ctrerr\fi}

\patchcmd{\l@section}{#1}{\textnormal{\uppercase{#1}}}{}{} % Разделы в оглавлении без выделения жирным, в верхнем регистре
\patchcmd{\l@section}{#2}{\textnormal{#2}}{}{} % Страницы без выделения жирным

\apptocmd{\appendix}{
    \renewcommand{\thesection}{\Asbuk{section}}
    \titleformat{\section}{\filcenter\fontsize{16pt}{16pt}\selectfont\bfseries}{}{0pt}{\MakeUppercase{\appendixname}~\thesection \\}{}{} % Изменение формата раздела приложения
    \renewcommand\thefigure{\Asbuk{section}.\arabic{figure}} % Изменении формата нумерации иллюстрации
    \renewcommand\thetable{\Asbuk{section}.\arabic{table}} % Изменении формата нумерации таблицы
    \renewcommand\theequation{\Asbuk{section}.\arabic{equation}} % Изменении формата нумерации формулы
    
    \let\oldsec\section
    \renewcommand{\section}{
        \clearpage
        \phantomsection
        \refstepcounter{section}
        \setcounter{figure}{0} % Счёт иллюстраций в пределах одного приложения
        \setcounter{table}{0} % Счёт таблиц в пределах одного приложения
        \setcounter{equation}{0} % Счёт иллюстраций в пределах одного приложения
        \addcontentsline{toc}{section}{\appendixname~\thesection}
        \oldsec*} % Нумерация раздела после названия
}

\makeatother

\usepackage[labelsep=endash]{caption} % Настройка пунктуации
\captionsetup[table]{justification=raggedright, singlelinecheck=false} % Выравнивание по левому краю надписи таблицы

\addto\captionsrussian{\renewcommand{\figurename}{Рисунок}} % Переопределение caption из babel


% Настройка заголовка списка использованных источников
\addto\captionsrussian{\renewcommand{\refname}{СПИСОК ИСПОЛЬЗОВАННЫХ ИСТОЧНИКОВ}}

\BeforeBeginEnvironment{thebibliography}{
    \phantomsection % для корректной ссылки в оглавлении 
    \makeatletter
    \renewcommand*{\@biblabel}[1]{#1.\hfill} % формат нумерации списка
    \makeatother}

\setlength{\bibindent}{-1.25cm} % Убрать отступы у элементов списка использованных источников + \{thebibliography}{99\kern\bibindent}


\begin{document}

\begin{titlepage}
    \pagestyle{empty}
    \setlength\parindent{0pt}
    \newcommand{\blankDate}[2]{\mbox{\uline{<<\makebox[.7cm]{#1}>>~\makebox[2cm]{#2}~\the\year{}~г.}}} % {день}{месяц}
    \newcommand\blankLine[2]{$\underset{\text{#1}}{\text{\uline{#2}}}$}
    \begin{center}
        \includegraphics[width=2.5cm]{MIREA_Gerb_Black} \par
        МИНОБРНАУКИ РОССИИ \par 
        Федеральное государственное бюджетное образовательное учреждение высшего образования \par
        \textbf{<<МИРЭА --- Российский технологический университет>>} \par
        \textbf{\fontsize{16pt}{16pt}\selectfont РТУ МИРЭА} \par
        \blankLine{(наименование института, филиала)}{Институт кибербезопасности и цифровых технологий} \par
        \blankLine{(наименование кафедры)}{Кафедра КБ-2 <<Прикладные информационные технологии>>} \par
        \begin{flushright}
            \begin{minipage}{.5\textwidth}
                \fontsize{12pt}{12pt}\selectfont
                \setlength{\parskip}{0pt}
                \centering
                Утверждаю \par
                Заведующий кафедрой КБ-2 \par
                \blankLine{(Ф.И.О.)}{И.~О.~Фамилия}~\blankLine{(подпись)}{\hspace{2cm}} \par
                \blankDate{}{}
            \end{minipage}
        \end{flushright}
        {\fontsize{16pt}{16pt}\selectfont
        \textbf{КУРСОВАЯ РАБОТА}} \par
        по дисциплине \blankLine{(наименование дисциплины)}{Языки программирования}
    \end{center}
    \textbf{Тема курсовой работы \uline{\hspace{11cm}}} \bigskip\par
    Студент группы \blankLine{учебная группа, фамилия, имя, отчество студента}{БСБО-05-20 В.~С.~Верхотуров\hspace{3cm}} \hfill\blankLine{подпись студента}{\hspace{3cm}} \bigskip\par
    Руководитель курсовой работы \blankLine{должность, звание, учёная степень}{\hspace{6cm}} \hfill\blankLine{подпись руководителя}{\hspace{3cm}} \bigskip\par
    Рецензент (при наличии) \blankLine{должность, звание, учёная степень}{\hspace{7cm}} \hfill\blankLine{подпись рецензента}{\hspace{3cm}} \bigskip\par
    \begin{tabular}{@{}ll}
        Работа предоставлена к защите & \blankDate{}{} \bigskip\\
        Допущен к защите & \blankDate{}{}
    \end{tabular}
    \begin{center}
        \vfill Москва~-- \the\year{}~г.
    \end{center}
    \newpage
    \textbf{Срок предоставления к защите курсовой работы до} \hfill\blankDate{}{} \par
    \textbf{Задание на курсовую работу выдал} \blankLine{(Ф.И.О. руководителя)}{\hspace{4cm}} \hfill\blankLine{(подпись руководителя)}{\hspace{3cm}} \par
    \hfill\blankDate{}{} \par
    \textbf{Задание на курсовую работу получил} \blankLine{(Ф.И.О. исполнителя)}{\hspace{3cm}} \hfill\blankLine{(подпись исполнителя)}{\hspace{3cm}} \par\bigskip
    \begin{center}
        Москва~-- \the\year{}~г.
    \end{center}
\end{titlepage}
\addtocounter{page}{2}

\begin{abstract}
    Этот документ имеет настройки, соответствующие у\-чеб\-но-ме\-то\-ди\-чес\-ко\-му пособию~\cite{bib:recomendations}, разделу 3, скомпилированный системой компьютерной вёрстки XeTeX. 
    
    В документе используются пакеты для форматирования документа и graphicx для иллюстраций. Полный список представлен в приложении~\ref{appendix:used_packages}. В родительской директории главного .tex файла должен лежать файл чёрно-белого герба для титульной страницы \verb"MIREA_Gerb_Black" (в шаблоне используется .eps файл~--- единственный векторный формат, предоставляемый на сайте вуза~\cite{bib:symbol}, из-за уязвимости .eps файла~\cite{bib:eps_cve} также возможно использование форматов JPEG, PNG). При удалении всего содержимого из окружения document настройки форматирования не изменятся.
    
    При использовании \url{https://overleaf.com} убедитесь, что в опциях проекта стоит компилятор XeLaTeX. 
    
    Замечания о расхождении с~\cite{bib:recomendations}, разделом 3,  можно писать в~\cite{bib:githubrepo}, Issues, задать вопрос в \url{https://t.me/ValerianaOfficinalis}.
    
    Примечание~--- документ скомпилирован \today{}
    
    \subsection*{.tex в .docx}
    
    В СМКО МИРЭА~\cite{bib:smko}, подразделе 1.1, рекомендуется использовать текстовый редактор, обеспечивающий корректное сохранение или экспорт документа в .doc (.docx). Шаблон .tex не может быть экспортирован в .doc (.docx). Возможно скомпилировать .pdf, сохранить в Google Документы и экспортировать в .docx или воспользоваться аналогичным конвертером. При этом настройки форматирования документа не сохраняются, возможен некорректный экспорт математических формул.
    
    \subsection*{Метаданные .pdf}
    
    Не забудьте в преамбуле в команде \verb"\hypersetup" поменять значение полей \verb"pdftitle={"Название моего документа\verb"}", \verb"pdfauthor={"Моё имя\verb"}".
    
\end{abstract}

\tableofcontents

\section*{Введение}
\phantomsection
\addcontentsline{toc}{section}{Введение}

Текст введения.

\section{Структура документа}

Структура документа совместима со стандартным классом документа article.

\subsection{Титульная страница}

Рекомендуется использовать выданные преподавателем титульные страницы (например, с помощью пакета pdfpages).
    
При использованих своих титульных страниц необходимо удалить окружение titlepage:
\begin{verbatim}
\begin{titlepage}
    ...
\end{titlepage}
\end{verbatim}
убрать опцию titlepage у класса документа:
\begin{verbatim}
\documentclass[14pt, a4paper]{extarticle} 
\end{verbatim}

К основному тексту после титульных листов необходимо добавить количество вставленных страниц. Пример:

\begin{verbatim}
\begin{titlepage}
    ...
\end{titlepage}
\addtocounter{page}{2}
\end{verbatim}

\subsection{Аннотация}

\begin{verbatim}
\begin{abstract}
    ...
\end{abstract}
\end{verbatim}

Подраздел аннотации:
\begin{verbatim}
\subsection*{...}
...
\end{verbatim}

Пункт аннотации:
\begin{verbatim}
\subsubsection*{...}
...
\end{verbatim}

\subsection{Оглавление}

\begin{verbatim}
\tableofcontents
\end{verbatim}

\subsection{Введение, раздел без нумерации}

\noindent\verb"\section*{"Введение\verb"}"\\
\verb"\phantomsection"\\
\verb"\addcontentsline{toc}{section}{"Введение\verb"}"\\
\verb"..."

\subsection{Раздел}

\begin{verbatim}
\section{...}
...
\end{verbatim}

Перед началом раздела в документ включаются все объявленные, но не отображённые плавающие окружения.

\subsection{Подраздел}

\begin{verbatim}
\subsection{...}
...
\end{verbatim}

Перед началом подраздела в документ включаются все объявленные, но не отображённые плавающие окружения.

\subsection{Пункт}

\begin{verbatim}
\subsubsection{}
...
\end{verbatim}

Перед началом пункта в документ включаются все объявленные, но не отображённые плавающие окружения.

\subsubsection{}

Пункты могут иметь только порядковый номер без заголовка.


\subsection{Список использованных источников}

\begin{verbatim}
\begin{thebibliography}{99\kern\bibindent}
    \bibitem{...} ...
    ...
\end{thebibliography}
\end{verbatim}

Ссылка на источник:
\begin{verbatim}
\cite{...}
\end{verbatim}

Пример ссылки на источник~\cite{bib:recomendations}.

За наличием ссылок и порядком элементов списка необходимо следить самостоятельно, либо использовать biblatex.


\subsection{Приложение}

Как и в стандартных классах перед приложениями необходимо указать команду \verb"\appendix".

Пример с одним приложением:
\begin{verbatim}
\appendix
\section{...}
...
\end{verbatim}

Пример с тремя приложениями:
\begin{verbatim}
\appendix
\section{...}
...
\section{...}
...
\section{...}
...
\end{verbatim}

Ссылка на приложение:
\begin{verbatim}
\ref{...}
\end{verbatim}

Пример ссылки на приложение~\ref{appendix:fig_tab_eq_numeration}.

За порядком приложений необходимо следить самостоятельно.

\subsection{Список}

\begin{verbatim}
\begin{itemize}
    \item ...,
    ...
\end{itemize}
\end{verbatim}

\begin{itemize}
    \item[] Пример простого списка:
    \item первый элемент,
    \item второй элемент.
\end{itemize}

\begin{itemize}
    \item[] Пример сложного списка:
    \item первый уровень вложенности;
    \begin{itemize}
        \item второй;
        \begin{itemize}
            \item третий;
            \begin{itemize}
                \item четвертый.
            \end{itemize}
        \end{itemize}
    \end{itemize}
\end{itemize}

\subsection{Перечисление}

\begin{verbatim}
\begin{enumerate}
    \item ...,
    ...
\end{enumerate} 
\end{verbatim}

\begin{enumerate}
    \item[] Пример простого перечисления:
    \item первый элемент,
    \item второй элемент.
\end{enumerate}

\begin{enumerate}
    \item[] Пример сложного перечисления:
    \item первый уровень вложенности,
    \begin{enumerate}
        \item второй;
        \begin{enumerate}
            \item третий;
            \begin{enumerate}
                \item четвертый;
                \item б;
                \item в;
                \item г;
                \item д;
                \item е;
                \item ж;
                \item и;
                \item к;
                \item л;
                \item м;
                \item н;
                \item о;
                \item п;
                \item р;
                \item с;
                \item т;
                \item у;
                \item ф;
                \item x;
                \item ц;
                \item ш;
                \item э;
                \item ю;
                \item я, при наличии б\'ольшего количества элементов компилятор выдаст ошибку.
            \end{enumerate}
        \end{enumerate}
    \end{enumerate}
\end{enumerate}


\subsection{Иллюстрация}
 
Пакет graphicx подключён.

\begin{verbatim}
\begin{figure}[htb]
    \centering
    \includegraphics[width=.9\textwidth]{...}
    \parskip=6pt
    \caption{...}
    \label{...}
\end{figure}
\end{verbatim}

См. рисунок~\ref{fig:test_label} на с.~\pageref{fig:test_label}, рисунок~\ref{fig:in_appendix} (в приложении).

\begin{figure}[htb]
    \centering
    \includegraphics[width=.5\textwidth]{MIREA_Gerb_Black}
    \parskip=6pt
    \caption{Подпись ниже рисунка по центру}
    \label{fig:test_label}
\end{figure}

Обратите внимание, что окружение figure является \emph{плавающим} в пределах раздела, и иллюстрация может появиться не там, где Вы ожидаете. Для размещения иллюстрации в конкретное место необходимо воспользоваться опцией H из пакета float (не подключён).

\subsection{Таблица}

См. таблицу~\ref{tab:test_label} на с.~\pageref{tab:test_label}, таблицу~\ref{tab:in_appendix} (в приложении).

\begin{verbatim}
\begin{table}[htb]
    \caption{...}
    \centering
    \begin{tabular}{|c|c|} 
        \hline
        1 & 2 \\ \hline
        3 & 4 \\ \hline
    \end{tabular}
    \label{...}
\end{table}
\end{verbatim}

\begin{table}[htb]
    \caption{Подпись над таблицей слева без абзацного отступа}
    \centering
    \begin{tabular}{ |c|c|c|c|c| } 
        \hline
        Ячейка 1 & Ячейка 2 & Ячейка 3 & Ячейка 4 & Ячейка 5 \\ \hline
        Ячейка 6 & Ячейка 7 & Ячейка 8 & Ячейка 9 & Ячейка 10 \\ \hline
    \end{tabular}
    \label{tab:test_label}
\end{table}

Обратите внимание, что окружение table является \emph{плавающим} в пределах раздела, и таблица может появиться не там, где Вы ожидаете. Для размещения таблицы в конкретное место необходимо воспользоваться опцией H из пакета float (не подключён).

\subsection{Уравнение и формула}

\begin{verbatim}
\begin{equation}
    a = b ,
\end{equation}\par
\end{verbatim}\vspace{-.5cm}
{где \verb"$a$~---" первая переменная; \verb"\\" \\
\verb"$b$~---" вторая переменная.}\bigskip

См. формулу~(\ref{eq:test_label}) в подразделе, формулу~(\ref{eq:in_appendix}) в приложении.

\begin{equation}\label{eq:test_label}
    \text{минус}\,a\times b=c ,
\end{equation}

где $a$~--- первая переменная; \\
$b$~--- вторая переменная; \\
$c$~--- третья переменная.

\BeforeBeginEnvironment{thebibliography}{\clearpage\phantomsection}


\begin{thebibliography}{99\kern\bibindent}
% Пример сервиса для форматирования библиографии https://open-resource.ru/spisok-literatury/
    \bibitem{bib:recomendations} Мерсов, А.А., Русаков, А.М., Филатов, В.В. Методические рекомендации по выполнению курсовой работы по дисциплине «Языки программирования».~--- М.: МИРЭА~--- Российский технологический университет, 2022.~--- 73 с.
    \bibitem{bib:symbol} Символика Университета // РТУ МИРЭА Режим доступа: \url{https://www.mirea.ru/mediapage/the-symbolism-of-the-university/}, свободный (дата обращения: 31.05.2022).
    \bibitem{bib:eps_cve} CVE-2013-4979 Detail // CVE Режим доступа: \url{https://www.cve.org/CVERecord?id=CVE-2013-4979}, свободный (дата обращения: 22.06.2022).
    \bibitem{bib:githubrepo} Шаблон XeTeX для курсовой работы по дисциплине <<Языки программирования>> // GitHub Режим доступа: \url{https://github.com/ValeryVerkhoturov/mirea-kb2-programming-languages}, свободный (дата обращения: 29.05.2022).
    \bibitem{bib:smko} СМКО МИРЭА 7.5.1/03.П.69-16 <<Рекомендации по оформлению письменных работ обучающихся по образовательным программам бакалавриата, специалитета, магистратуры>> от 26.10.2016 2016
\end{thebibliography}


\appendix

\section{Используемые пакеты}
\label{appendix:used_packages}

\begin{itemize}
    \item babel,
    \item caption,
    \item fontspec,
    \item geometry,
    \item graphicx,
    \item hyperref,
    \item indentfirst,
    \item newtxmath,
    \item placeins,
    \item titlesec,
    \item tocloft,
    \item ulem.
\end{itemize}

\section{Нумерация иллюстраций и таблиц в приложении}
\label{appendix:fig_tab_eq_numeration}

\begin{figure}[htb]
    \centering
    \includegraphics[width=.5\textwidth]{MIREA_Gerb_Black}
    \parskip=6pt
    \caption{Иллюстрация в приложении}
    \label{fig:in_appendix}
\end{figure}

\begin{table}[htb]
    \caption{Таблица в приложении}
    \centering
    \begin{tabular}{ |c|c|c|c|c| } 
        \hline
        Ячейка 1 & Ячейка 2 & Ячейка 3 & Ячейка 4 & Ячейка 5 \\ \hline
        Ячейка 6 & Ячейка 7 & Ячейка 8 & Ячейка 9 & Ячейка 10 \\ \hline
    \end{tabular}
    \label{tab:in_appendix}
\end{table}

\begin{equation}\label{eq:in_appendix}
    \text{минус}\,a\times b=c ,
\end{equation}

где $a$~--- первая переменная; \\
$b$~--- вторая переменная; \\
$c$~--- третья переменная.


\end{document}
